\documentclass[a4paper,12pt]{article}
\usepackage[T1]{fontenc}
\usepackage{inputenc}
\usepackage{fontspec}
\usepackage{lmodern}
\usepackage[english]{babel}
\usepackage{hyperref}
\usepackage{float}
\usepackage{soul} % pour surligner le texte
\usepackage{tikz} % pour la définition de couleur

% MISE EN FORME DES BLOCS DE CODE
\usepackage{listings} % pour les blocs de code

\definecolor{whitegray}{gray}{0.97}
\definecolor{pearlgray}{gray}{0.94}
\definecolor{lightgray}{gray}{0.90}
\definecolor{darkgray}{gray}{0.50}
\definecolor{green}{rgb}{0.22, 0.73, 0.56}
\definecolor{red}{rgb}{0.8, 0.25, 0.33}
\definecolor{blue}{rgb}{0.36, 0.62, 0.92}
\definecolor{bluegray}{rgb}{0.33, 0.41, 0.47}

% Pour que le texte en police console soit surligné en gris clair
\let\OldTexttt\texttt
\sethlcolor{pearlgray}
\renewcommand{\texttt}[1]{\OldTexttt{\hl{#1}}}

% Pour la colorisation des nombres dans le code seulement quand il ne font pas partie de chaine de caractères ou commentaire
\newcommand\digitstyle{\color{blue}}
\makeatletter
\newcommand{\ProcessDigit}[1]
{%
	\ifnum\lst@mode=\lst@Pmode\relax%
	{\digitstyle #1}%
	\else
	#1%
	\fi
}

% Definition de l'apparence des blocs de code
\lstset{
	language=php, % définition du langage (php et javascript ont des syntaxes proches)
	basicstyle=\scriptsize\ttfamily, % petite taille, police de console
	stringstyle=\color{red}, % chaine de caractères
	backgroundcolor=\color{whitegray}, % fond
	commentstyle=\color{darkgray}\textit, % commentaires
	numberstyle=\tiny\color{darkgray}, % numero de ligne
	rulecolor=\color{lightgray}, % bordure
	keywordstyle=\color{green}\textbf, % opérateurs logiques etc
	literate=
	{0}{{{\ProcessDigit{0}}}}1
	{1}{{{\ProcessDigit{1}}}}1
	{2}{{{\ProcessDigit{2}}}}1
	{3}{{{\ProcessDigit{3}}}}1
	{4}{{{\ProcessDigit{4}}}}1
	{5}{{{\ProcessDigit{5}}}}1
	{6}{{{\ProcessDigit{6}}}}1
	{7}{{{\ProcessDigit{7}}}}1
	{8}{{{\ProcessDigit{8}}}}1
	{9}{{{\ProcessDigit{9}}}}1
	{<=}{{\(\leq\)}}1,
	morestring=[b]",
	morestring=[b]',
	morecomment=[l]//,
	aboveskip=3mm,
	belowskip=-2mm,
	breakatwhitespace=false,
	breaklines=true,
	captionpos=b,
	deletekeywords={...},
	escapeinside={\%*}{*)},
	extendedchars=true,
	framexleftmargin=16pt,
	framextopmargin=5pt,
	framexbottommargin=5pt,
	frame=tblr,
	keepspaces=true,
	morekeywords={*,...},
	numbers=left,
	numbersep=10pt,
	showspaces=false,
	showstringspaces=false,
	showtabs=false,
	stepnumber=1,
	tabsize=4,
	title=\lstname,
}

\begin{document}

\section{ElasticSearch query language}\label{elasticsearch-query-language}

ElasticSearch has its own query language, more documentation can be
found \href{https://www.elastic.co/guide/en/elasticsearch/reference/current/query-dsl.html}{here}

\subsection{Anatomy of a query}\label{anatomy-of-a-query}

In most cases, an ElasticSearch query is composed of :
\begin{itemize}
	\item a \textbf{header} defining~:
	\begin{itemize}
		\item which \textbf{\emph{method}} is to be used~: \texttt{GET}, \texttt{POST}, etc.
		\item which \textbf{\emph{index}} (i.e.~which entity of the database written in snake\_case) is going to be queried. Not necessary if all indexes will be queried.
		\item what \textbf{\emph{type}} of search is to be made (\texttt{\_search} most ofthe time)
	\end{itemize}
	\item a \textbf{body} defining (among other things)~:
	\begin{itemize}
		\item the \textbf{\emph{fields}} that are going to appear in the results (\texttt{\_source})
		\item the \textbf{\emph{filters}} that are going to narrow the number of results matching those filters (\texttt{query})
		\item different properties of the results (size of the results, index from which to begin, etc.)
	\end{itemize}
\end{itemize}

\paragraph{Tips \& tricks}\label{tips-tricks}

The body of a query needs to be a correctly formatted JSON string, even using single quotes instead of double is considered to be an error. The \texttt{dev tools} tab in Kibana interface offers help for automatic indentation and autocompletion features that can be very handy.

\subsection{Get all records}\label{get-all-records}

To get all the records of the entire database :

\begin{lstlisting}
GET _search
\end{lstlisting}

To retrieve all records from an index (primary source in this case) :

\begin{lstlisting}
GET primary_source/_search
\end{lstlisting}

Which is equivalent to :

\begin{lstlisting}
GET primary_source/_search
{
	"query": {
		"match_all": {}
	}
}
\end{lstlisting}

\subsection{Simple \href{https://www.elastic.co/guide/en/elasticsearch/reference/current/query-dsl-match-query.html}{matching query}}\label{simple-matching-query}

\paragraph{Exact term match}\label{exact-term-match}

\begin{quote}
All the works that contain the string ``tabule'' in their title.
\end{quote}

Note that the case of the letters doesn't matter, it will match ``Tabule'' as well. In addition to that, in this configuration, a sub-string will not match the same as the entire string (``tabu'' will not match ``tabule'') ; the string is treated as a complete word.

\begin{lstlisting}
GET work/_search
{
    "query":{
        "match": {
            "title": "tabule"
        }
    }
}
\end{lstlisting}

\begin{quote}
All the original items that are associated with a primary source that is kept in the library that have \texttt{2} as id.
\end{quote}

\begin{lstlisting}
GET original_text/_search
{
    "query": {
        "match": {
            "primary_source.library.id": "2"
        }
    }
}
\end{lstlisting}

\paragraph{Multiple terms match}\label{multiple-terms-match}

\begin{quote}
All original items that have in their title either the word ``solis'', either the word ``lune''.
\end{quote}

In this configuration, each string separated by a space is treated individually and the operator used to connect them is \texttt{OR}. In other words, the more you put terms, the more you will match original items.

\begin{lstlisting}
GET original_text/_search
{
    "query":{
        "match": {
            "original_text_title": "solis lune"
        }
    }
}
\end{lstlisting}

To get a response where each terms given are independently going to filter the result (same kind of behavior as a Google query), you need to specifies the operator to be \texttt{AND}.

\begin{lstlisting}
GET original_text/_search
{
    "query": {
        "match": {
            "original_text_title": {
                "query": "lune solis",
                "operator": "AND"
            }
        }
    }
}
\end{lstlisting}

\subsection{Adding some margin of error}\label{adding-some-margin-of-error}

\paragraph{Fuzziness}\label{fuzziness}

The fuzziness allows a certain amount of inaccuracy to be accepted.

\begin{quote}
All library that approximately have the string ``natonale'' in their name.
\end{quote}

\begin{lstlisting}
{
    "query": {
        "match": {
            "library_name": {
                "query": "natonale",
                "fuzziness": "auto"
            }
        }
    }
}
\end{lstlisting}

You can set the fuzziness to \texttt{1} or more but the \texttt{auto} settings allows a number of letters that do not match, proportional to the length of the term to be searched.

\subsubsection{Full text search on an index}\label{full-text-search-on-an-index}

To allow search on every field of an entity, the query has to be set to \texttt{multi\_match}.

\begin{quote}
All edited texts that have approximately the string ``lune'' in one of them fields.
\end{quote}

\begin{lstlisting}
GET edited_text/_search
{
    "query": {
        "multi_match": {
            "query": "lune",
            "fuzziness": "auto"
        }
    }
}
\end{lstlisting}

As is, those kind of requests are deprecated because no fields are specified : ElasticSearch encourages to list the fields you want the query to be executed on. The \texttt{fields} allows to reduce noise in the results and to take less time.

\begin{quote}
All primary sources that match approximately the strings ``vatican'' and ``latin'' in the list of fields specified.
\end{quote}

\begin{lstlisting}
GET primary_source/_search
{
    "query": {
        "multi_match": {
            "query": "vatican latin",
            "fuzziness": "auto",
            "operator": "and",
            "fields": [
                "shelfmark",
                "digital_identifier",
                "kibana_name",
                "tpq.keywork",
                "taq.keywork",
                "prim_type",
                "library.kibana_name",
                "original_texts.kibana_name",
                "original_texts.table_type.kibana_name",
                "original_texts.place.kibana_name",
                "original_texts.historical_actor.kibana_name",
                "original_texts.script.script_name",
                "original_texts.language.language_name"
            ]
        }
    }
}
\end{lstlisting}

Notice that on the fields \texttt{tpq} and \texttt{taq} that are typed as integer, a string query cannot be performed. In order to query those fields as well, you must add \texttt{.keyword} after the field name : it corresponds to the field but typed as a string.

\paragraph{Wildcards}\label{wildcards}

To find some more documentation for wildcard queries, click \href{https://www.elastic.co/guide/en/elasticsearch/reference/current/query-dsl-wildcard-query.html}{here}.

\subsection{Defining the source}\label{defining-the-source}

If you are not interested in all the metadata (i.e.~the content of the fields) associated with the entity you want to query, it is possible to set a list of fields that are going to appear in the response.

\begin{quote}
Only the shelfmark, and the library name of all primary sources that are a manuscript
\end{quote}

\begin{lstlisting}
GET primary_source/_search
{
    "_source": [
        "shelfmark",
        "library.library_name"
    ],
    "query": {
        "match": {
            "prim_type": "ms"
        }
    }
}
\end{lstlisting}

Note that if a record in the result do not have some information you asked for (let's say, the primary source isn't associated with a library, thus doesn't have a \texttt{library.libray\_name}), the result object will not have the key for this precise field. Instead of looking like that :

\begin{lstlisting}
"_source" : {
    "library" : {
        "library_name" : "Vatican Library"
    },
    "shelfmark" : "Vat. Pal. Lat. 1376"
}
\end{lstlisting}

It will look like :

\begin{lstlisting}
"_source" : {
    "shelfmark" : "Vat. Pal. Lat. 1376"
}
\end{lstlisting}

\subsection{Special queries}\label{special-queries}

\subsubsection{\href{https://www.elastic.co/guide/en/elasticsearch/reference/current/query-dsl-range-query.html}{Range queries}}\label{range-queries}

Range queries can be made on fields that are numbers (even if the field is typed as a string but contains integer) in the Kibana interface, but does seem to only work on integer/float/date typed field when using ajax.

\begin{quote}
All the primary source that have an edition date between 1400 and 1500
\end{quote}

\begin{lstlisting}
GET primary_source/_search
{
    "query": {
        "range": {
            "date": {
                "gte": 1400, // greater than
                "lte": 1500 // less than
            }
        }
    }
}
\end{lstlisting}

If you want to make a range query on a date typed field (fields that end with \texttt{\_date}, you can use some ElasticSearch \href{https://www.elastic.co/guide/en/elasticsearch/reference/current/common-options.html\#date-math}{tools for date math} (those as well, seems to cause problem when used with ajax) :

\begin{quote}
All original items that have been created between 1000 years before today and 25 years after 1500
\end{quote}

\begin{lstlisting}
GET original_text/_search
{
    "query": {
        "bool": {
            "must": [
                {
                    "range": {
                        "tpq_date": {
                            "gte":"now-1000y"
                        }
                    }
                },
                {
                    "range": {
                        "taq_date": {
                            "lte": "1500-01-01||+25y"
                        }
                    }
                }
            ]
        }
    }
}
\end{lstlisting}

\subsubsection{\href{https://www.elastic.co/guide/en/elasticsearch/reference/current/query-dsl-geo-distance-query.html}{Geo-distance} queries}\label{geo-distance-queries}

The fields named \texttt{location} holds information that is treated as geo point by ElasticSearch : \texttt{geo\_distance} queries can be executed on them.

\begin{quote}
All works that have been conceived around 100km from 48 of latitude and 2 of longitude. \textbf{NB} : in this syntax, \texttt{longitude} comes before \texttt{latitude}.
\end{quote}

\begin{lstlisting}
GET work/_search
{
    "query": {
        "geo_distance": {
            "distance": "100km",
            "place.location": [2,48]
        }
    }
}
\end{lstlisting}

The same query can be formulated more explicitly with this syntax :

\begin{lstlisting}
GET work/_search
{
    "query": {
        "geo_distance": {
            "distance": "100km",
            "place.location": {
                "lat" : 48,
                "lon" : 2
            }
        }
    }
}
\end{lstlisting}

\subsection{Combining multiple clauses}\label{combining-multiple-clauses}

Every filter you want to combine to build a query can be add with this kind of structure :

\begin{lstlisting}
{
    "query": {
        "bool": {
            "must": [
                {
                    // filter 1
                },
                {
                    // filter 2
                }
            ]
        }
    }
}
\end{lstlisting}

\paragraph{Putting all together}\label{putting-all-together}

\begin{quote}
The shelfmarks of all early printed primary sources that contains an original item that were created near Paris (lat : 48, long : 2).
\end{quote}

\begin{lstlisting}
GET original_text/_search
{
    "_source": [
        "primary_source.shelfmark"
    ],
    "query": {
        "bool": {
            "must": [
                {
                    "geo_distance": {
                        "distance": "100km",
                        "place.location": [2,48]
                    }
                },
                {
                    "match": {
                        "primary_source.prim_type": "ep"
                    }
                }
            ]
        }
    }
}
\end{lstlisting}
\end{document}
